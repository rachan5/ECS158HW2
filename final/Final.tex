\documentclass[11pt,letterpaper]{article}

\usepackage[margin=1in]{geometry}
\usepackage{amsthm}
\usepackage{amssymb}
\usepackage{listings}
\usepackage{color}
%\usepackage{bbm}
%\usepackage{url}

\definecolor{mygreen}{rgb}{0,0.6,0}
\definecolor{mygrey}{rgb}{0.5,0.5,0.5}

\lstdefinestyle{numbers} {numbers=left, stepnumber=1, numberstyle=\tiny, numbersep=10pt}
\lstdefinestyle{MyFrame}{backgroundcolor=\color{yellow},frame=shadowbox}

\lstdefinestyle{MyC++Style} {language=C++,style=numbers,style=MyFrame,frame=none,backgroundcolor={}}
\lstdefinestyle{MyR}{
	backgroundcolor = \color{white},
	commentstyle  = \color{mygreen},
	frame = single,
	keywordstyle = \color{blue},
	language = R,
	numbers = left,
	numbersep = 8pt,
	numberstyle = \color{mygrey},
}
\lstset{language=C++,frame=none}
\lstset{language=R,frame=none}

% Use this space to define/redefine commands.  This is a really useful trick if there's some 
% command sequence that you use quite frequently that you don't want to have to type over and over.
% For example, here I define new commands for floor and ceiling.  The syntax is
% 
% \newcommand{\commandname}[numberofargs]{stuff here}
%
% Use #1, #2, etc. in the "stuff here" section to refer to your arguments.  Here I define commands
% for "floor" and "ceiling"

\newcommand{\floor}[1]{{\left\lfloor #1 \right\rfloor}}
\newcommand{\ceil}[1]{{\left\lceil #1 \right\rceil}}

% Here I'm defining a theorem to be used later.  The syntax is
% \newtheorem{theoremname}{Text}

\newtheorem{thm1}{Theorem}

% I don't like the default proof symbol for the theorems, so I made my own.  (You can use this 
% command to redefine any LaTeX command you want.  Be careful, though!  This may break things!
% The syntax is identical to the \newcommand macro.)
\renewcommand{\qedsymbol}{$\scriptstyle\blacksquare$}

% All of your content needs to go between \begin{document}...\end{document}
\begin{document}

% You can use \begin{flushright}...\end{flushright} to right-justify your text, as well.  To change
% text size, choose one of (in decreasing order) \Huge, \huge, \LARGE, \Large, \large, \small,
% \footnotesize, \tiny.  By default, your text is \normalsize.
\begin{center}
{\huge \textbf{ECS 158 Final Project: An Attempt At Parallelizing R's Phylobase::ShortestPath()}\\
\vspace{5mm}
\begin{Large}
Raymond S. Chan, Alicia Luu, Bryan Ng
997544611, 999999999, 999999999\\
raschan@ucdavis.edu, ajuu@ucdavis.edu, bng@ucdavis.edu\\
\end{Large}}
\end{center}

% You can create sections, subsections, and subsubsections this way.  You can \label them and
% reference later in the document:

\vspace{5mm}

\begin{center}
	\begin{large}
		\textbf{Abstract}\\
		This report attempts paralleize CRAN's phylobase package's shortestPath function in RSnow, OpenMP and CUDA. The function takes in a phylogeneic tree, and produces the shortest path between two nodes. The RSnow implementation is built on top of the existant code and parallelizes the descendants function. The OpenMP implementation did a similar approach, by parallelizing the ancestors and descendants C functions. The CUDA implementation took the brute force approach to the  shortestPath problem. Overall, \textbf{< SENTENCE ABOUT TEST RESULTS >}. \textbf{< CLOSE WITH A MORE GENERAL STATEMENT BUILDING OFF RESULTS OF THE TESTS >} 
	\end{large}
	

\end{center}

\section{Introduction and Motivation}
 
Alongside the rise of "big data" in the recent years, bioinformatices has gained considerable momentum. But the a consistent issues remain: what do we do with all the data and how do we make sense of it at a reasonable rate? The R community has taken a stab at those issues. For this report, we are examining the R's phylobase package, which provides the base class and functions for phylogenetic or evolutionary structures and comparitive data. (CITE) We will be focusing our efforts in making the "treewalk" utility functions, such as finding descendants/ancestors and shortest pathes, fast through three different parallel programming models (RSnow, OpenMP and CUDA).  


\section{Approaches}

Only pseudocode or key chunks of code of each implementation are shown or described,
see Appendix A for code details.

\subsection{Original}
The original R implementation calculated the shortest path betweeen the two nodes of interest by first calculating their Most Recent Common Ancestor (MRCA). Then that MRCA's descendants are calculated and compared to the the two nodes of interest's ancestors. Any overlap is stored and that is the shortest path. \\
The C version of the descendants function, called Cdescendant(), works by first marking given node in a preordered list of edges. Then the direct descendants (or children) of the given node is marked. Cdescendant() then iterates through the other edges and marks each marked node's direct descendants (children).\\
The C version of the ancestors function works the same as Cdescendants, expect direct \textit{ancestors} (or parents) are marked instead of direct \textit{descendants} (children).

Pseudocode (source code in original phylobase package):
\begin{lstlisting}[style=MyR]
descendants(tree, given_node){
	#let x be the edges of tree listed in PREORDER,
	#with the older node occupying the first column
	# C function call
	isDescendant <- Cdescendants(x[,1], x[,2], given_node)
	retval <- getNode(tree, isDescendant)
}

ancestors(tree, node1){
	#let x be the edges of tree listed in POSTORDER,
	#with the older node occupying the first column
	# C function call
	isAncestor <- Cancestors(x[,1], x[,2], given_node)
	retval <- getNode(tree, isAncestor)
}

MRCA(tree, node1, node2 ... noden){
	nodes <- unique(node1, node2, ..., noden)
	ancests <- lapply(nodes, ancestors, phy=phy, type="ALL")
    retval <- getNode(phy, max(Reduce(intersect, ancests)))
}
shortestPath(tree, node1, node2){
     t1 <- getNode(tree, node1)
     t2 <- getNode(tree, node2)

     # most recent common ancestor
     comAnc <- MRCA(tree, t1, t2) 
     desComAnc <- descendants(tree, comAnc)

     # path: common ancestor to t1
     ancT1 <- ancestors(x, t1)
     path1 <- intersect(desComAnc, ancT1) 

     # path: common ancestor to t2
     ancT2 <- ancestors(x, t2)
     path2 <- intersect(desComAnc, ancT2)

     # union of the path above paths
     retval <- union(path1, path2)
}
\end{lstlisting}


\subsection{RSnow}
The RSnow implemenation builds on top of original R version by paralleizing the descendants function. 
In order to make independent subproblems, for a given node, every other node keeps marching upwards to its ancestors until they either reached the root or encountered the given node. 
We were unable to parallelize the ancestor function. The original R version seems to have taken the most efficent serial approach for calculating a given node's ancestors. \\

Pseudocode: (see code in Appendix A.1)
\begin{lstlisting}[style=MyR]
descendants(tree, given_node){
     #let x be the list of all nodes except the given_node
     for i from 1 to height of tree
          if x == given_node -> append node to retval
          update x with its correponding ancestor
     return retval
}
\end{lstlisting}

\subsection{OpenMP}
Because OpenMP allows parallelism to be added without changing a significant amount of existing code, we implement OpenMP code with minimal changes to the original phylobase program. 
We decided to parallelize all serial C functions called and implied by the “shortest path” function in treewalk.R
There are only two C functions implicitly called by the shortestPath function: one in ancestors.c and one in descendants.c. 
Because of the serial “for” loops which comprised most of the ancestor and descendant C functions, the OpenMP implementation is embarrassingly parallel. The functions are also small so there isn't a need for explicit barriers or single/master calls.
\\

See code in Appendix A.2.\\

\subsection{CUDA}
Our CUDA implementation of the shortestPath function utilizes the GPU to find all ancestors of a given pair of nodes and then construct the shortest path between them. 
Our implementation assumes CSIF's pc43's resources, which are 1024 threads per block and 1 GB of global memory. We assume the given data can fit in our GPU’s global memory. 
This assumption may limit the test we will be able to perform. 
Our solution utilizes the fact that the shortest path between two nodes in a tree must converge at the lowest common ancestor of both nodes. 
In cases, where one node is an ancestor of another, then the shortest path is then found by traversing the parents of the child node. 
We parallelized our code by finding both sets of ancestors of the given nodes at the same time. 
Since neither node needs to know about the other to find its own ancestors, this problem can be done independently of each other. 
Both sets of ancestors are then traversed to find the shortest path. 
We were unable to parallelize this part of the solution since each list of ancestors must be checked to find overlapping elements. \\

See code in Appendix A.3.\\

\section{Experiment Results}

These are the results of running the above scripts with a simplified internet (i.e.  n = 6).
\begin{verbatim}
>> i = [ 2 6 3 4 4 5 6 1 1];
>> j = [ 1 1 2 2 3 3 3 4 6];
>> n = 6;
>> G = sparse(i,j,1,n,n);
>> Finaltimetest
pagerank1          pagerank2          pagerank3A         pagerank3B        pagerankpow
---------------------------------------------------------------------------------------
  0.0002            0.0001             0.0002            0.0003            0.0004
\end{verbatim}
These are the results of running the above scripts with the Harvard500 dataset (i.e. n = 500).
\begin{verbatim}
>> load Harvard500
>> Finaltimetest
pagerank1          pagerank2          pagerank3A         pagerank3B        pagerankpow
---------------------------------------------------------------------------------------
  0.0024            0.0329             0.0226            0.0011            0.0255
\end{verbatim}

These results are consistent with the dicussion above. 

\section{Discussion}

\subsection{RSnow}
TALK ABOUT BIG O's.\\

\subsection{OpenMP}
TALK ABOUT BIG O's.\\

\subsection{CUDA}

\textbf{THIS IS JUST A GUESS.}\\ Compared to the serial version, the cuda implementation performed slower in most test cases. While the cuda version can compute both given node’s ancestors at the same time, it must also load the entire tree into the GPU’s memory.\\

\section{Conclusion}

The PageRank algorithm was the starting point of Google's rise to fame. It was able to numerically quantify the "quality" on links/web pages of the internet. Pagerank is a Markov chain for which we solve for the dominant eigenvector of its transition probability matrix. There are two main methods of solving such a system of linear equations. The Power method is shown here to be the best method because of its efficiency in run time and memory usage. The run times are decent, as Hopcraft stated, it varies logarithmically with the size of input (n web paes). For the current day, the sheer amount of data that needs to processed is daunting. Further studies on PageRank could be done in further optimizing its space usage. 

\section{Acknowledgements}
We would like to thank Professor Norman Matloff for his guidance and knowledge presented during lectures. This work is the result of a final project for ECS 158 Winter Quarter 2015.  His open-source textbook, blog and various tutorial were an essential part of our learning. We would like to thank the teaching assistance Shengren Li for offering invaluble advice and feedback on our codes (especically our CUDA) throughout the quarter.

\section{Appendix}

\appendix
\section{Codes} \label{App:AppendixA}
% the \\ insures the section title is centered below the phrase: AppendixA

INSERT ALL CODES HERE ALONG WITH A PARAGRAPH EXPLAINING IT
\subsection{RSnow Code}
\begin{lstlisting}[style=MyR]
SNOW <- function(x,size,root,type=c("descendants")){
    ans <- rep(0,size)
    mystart <- (myid-1)*length(x)+1
    myend <- myid*length(x)

    type <- match.arg(type)
    if (type == "descendants"){
        v1 <- descendant
        v2 <- ancestor
        #initalization
        temp <- v1[mystart:myend]

        #second and beyond iteration
        for (j in 1:size){
            if (node %in% temp){
                setthese <- which(temp == node) + mystart-1
                ans[setthese] <- 1
            }
            blah <- rep(-1,length(temp))
            for (i in (1:length(temp))){
                matched_pos <- which(v1 == temp[i])
                if (length(matched_pos) != 0){
                    blah[which(temp == temp[i])] <- matched_pos
                }
                else{#matched_pos == 0
                    ## R is 1 INDEXED!
                    if (type == "descendants"){
                        blah[i] <- 1
                    }
                }
            }#for i 
            #"go to your parents set"
            difference <- length(temp) - length(v2[blah])
            temp <- v2[blah]
            if (difference > 0){
                temp <- c(rep(0,difference),temp)
            }
            if (node %in% temp){
                setthese <- which(temp == node) + mystart-1
                ans[setthese] <- 1
            }
        }#j loop
    }#new endif for type==descendants
    return(ans)
}# end SNOW

setmyid <- function(i){
    myid <<- i
}

## get descendants with RSnow
RSnowdescendants <- function (phy, node, type=c("tips","children","all"),cls) {
    type <- match.arg(type)

    ## look up nodes, warning about and excluding invalid nodes
    oNode <- node
    node <- getNode(phy, node, missing="warn")
    isValid <- !is.na(node)
    node <- as.integer(node[isValid])

    if (type == "children") {
        res <- lapply(node, function(x) children(phy, x))
        ## if just a single node, return as a single vector
        if (length(res)==1) res <- res[[1]]
    } else {
        ## edge matrix must be in preorder for the C function!
        #if (phy@order=="preorder") {
            edge <- phy@edge
        #} else {
        #    edge <- reorder(phy, order="postorder")@edge
        #}
        ## extract edge columns
        ancestor <- as.integer(edge[, 1])
        descendant <- as.integer(edge[, 2])

        ## return indicator matrix of ALL descendants (including self)
        #isDes <- .Call("descendants", node, ancestor, descendant)
        clusterExport(cls,c("node", "ancestor", "descendant","setmyid","SNOW"), envir=environment())
        dexgrps <- splitIndices(length(ancestor),length(cls))
        rootdex <- which(phy@edge[,1] == 0)
        clusterApply(cls,1:length(cls),setmyid)
        newisDes <- clusterApply(cls,dexgrps,SNOW,length(ancestor),rootdex,     "descendants")
        isDes <- (matrix(Reduce('+',newisDes),nrow=length(ancestor),ncol=1))
        storage.mode(isDes) <- "logical"
        ## for internal nodes only, drop self (not sure why this rule?)
        int.node <- intersect(node, nodeId(phy, "internal"))
        isDes[cbind(match(int.node, descendant),
            match(int.node, node))] <- FALSE

        ## if only tips desired, drop internal nodes
        if (type=="tips") {
            isDes[descendant %in% nodeId(phy, "internal"),] <- FALSE
        }
        ## res <- lapply(seq_along(node), function(n) getNode(phy,
        ##     descendant[isDes[,n]]))
        res <- getNode(phy, descendant[isDes[, seq_along(node)]])
    }
    ## names(res) <- as.character(oNode[isValid])

    res
}

###############
# shortestPath
###############

RSnowshortestPath <- function(phy, node1, node2,cls){

    ## conversion from phylo, phylo4 and phylo4d
    if (class(phy) == "phylo4d") {
        x <- extractTree(phy)
    }
    else if (class(phy) != "phylo4"){
        x <- as(phy, "phylo4")
    }
    ## some checks
    t1 <- getNode(x, node1)
    t2 <- getNode(x, node2)
    if(any(is.na(c(t1,t2)))) stop("wrong node specified")
    if(t1==t2) return(NULL)

    ## main computations
    comAnc <- MRCA(x, t1, t2) # common ancestor
    desComAnc <- RSnowdescendants(x, comAnc, type="all",cls)
    ancT1 <- ancestors(x, t1, type="all")
    path1 <- intersect(desComAnc, ancT1) # path: common anc -> t1

    ancT2 <- ancestors(x, t2, type="all")
    path2 <- intersect(desComAnc, ancT2) # path: common anc -> t2

    res <- union(path1, path2) # union of the path
    ## add the common ancestor if it differs from t1 or t2
    if(!comAnc %in% c(t1,t2)){
        res <- c(comAnc,res)
    }

    res <- getNode(x, res)

    return(res)
} # end shortestPath

\end{lstlisting}

\subsection{OpenMP Code}
\begin{lstlisting}[style=MyC++Style]

\end{lstlisting}

\subsection{CUDA Code}
\begin{lstlisting}[style=MyC++Style]
BRYAN CODE GOES HERE
\end{lstlisting}

\section{Who Did What} \label{App:AppendixB}
% the \\ insures the section title is centered below the phrase: Appendix B

Alicia wrote the OpenMP implementation. Bryan wrote the CUDA implemenation. Raymond wrote the RSnow implementation. We worked on running tests and writing the report in  \LaTeX.

\newpage

\begin{thebibliography}{1}

  \bibitem{Moler} C. Moler, {\em Numerical Computing with MATLAB Revised Reprint}  2004.

  \bibitem{Page}  L. Page, S. Brin, R. Motwani, and T. Wingograd, {\em The PageRank Citation Ranking: Bringing Order to the Web}, avaiable at http://ilpubs.stanford.edu:8090/422/1/1999-66.pdf,	1998

  \bibitem{Hopcraft}  J. Hopcraft and R. Kannan, {\em Foundations of Data Science}, available at http://www.cs.cornell.edu/jeh/NOSOLUTIONS90413.pdf, 2011

\end{thebibliography}

\iffalse
\subsection{Other Useful Tricks}
You already saw an example of formatting summations, so here's another example of product notation
(fun exercise: assume $a,b,c...,z$ are constant.  What does the following product equal?):

% Sometimes you have to manually specify your page breaks
\pagebreak

The other useful way you can use the array syntax is to line up a sequence of equations or
inequalities:

% This is so common that there is a special environment defined for it.  
%The eqnarray environment will number your equations by default, so you can reference them later.  
%If you don't want this to happen, use the \begin{eqnarray*}...\end{eqnarray*} environment instead.  

% The syntax is a little bit easier here; you only get three columns, and you don't have any control
% over the alignment.  You still use & to move between columns and \\ to add new rows.  Note that
% in both the eqnarray and array environments, you don't have to put anything in a particular cell.
\begin{eqnarray}
x^2 + 5x + 6 & = & x^2 + 2x + 3x + 6\\
& = & x(x + 2) + 3(x + 2)\label{eqn2}\\
& = & (x + 3)(x + 2)
\end{eqnarray}

If you're trying to write a formal theorem and a proof, you can do so in the following way (note
that you must define the theorem in your document header, much like you would define a new command):

% We defined this theorem on line 25.  If you want to cite your theorem as proved by someone else,
% do so in the square brackets following):
\begin{thm1}[Fermat]\label{thm1}
Let $n > 2$ be an integer.  There are no positive integers $a,b,c$ that satisfy the following 
equation:
\[a^n + b^n = c^n\]
\end{thm1}
\begin{proof}
This is trivial!  Go do something hard, like prove $P = NP$.
\end{proof}

\fi
\end{document}


